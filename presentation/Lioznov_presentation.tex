\documentclass[8pt,pdf,hyperref={unicode},serif]{beamer}

\usepackage{amsfonts}
\usepackage{amsmath}
\usepackage{amssymb}
\usepackage{booktabs} %пакет для использования \specialrule{}{}{} в \tabular{}{}
\usepackage{caption}
\usepackage{cite} 
\usepackage{color}
\usepackage{delarray}
\usepackage{enumerate}
\usepackage{fancyhdr} 
\usepackage{float}
\usepackage{indentfirst} %отступ в первом абзаце любого раздела
\usepackage{mathtext}
\usepackage[section]{placeins}
\usepackage{textcomp}
\usepackage{upgreek}
\usepackage{wrapfig}

\usepackage[english,russian]{babel}
\usepackage[utf8]{inputenc} %чтобы можно было набирать русский текст
\usepackage[T2A]{fontenc} %кодировка текстовых шрифтов с поддержкой кириллических букв
\usepackage{cmap}
\usepackage{listingsutf8} 

\graphicspath{{./images/}}
\usepackage{svg}
\renewcommand{\theequation}{\arabic{section}.\arabic{equation}}
\setsvg{inkscape = inkscape -z -D}
\setsvg{svgpath = images/}
\usepackage{epstopdf}

% отключить клавиши навигации
\setbeamertemplate{navigation symbols}{}

% тема оформления
\usetheme{CambridgeUS}

% цветовая схема
\usecolortheme{seahorse}

\title{Численное моделирование ускорения частиц на ударных волнах } % заголовок
\author{Антон Лиознов} % авторы
\date{22 июня 2015} % день (сегодня)
\institute{Санкт-Петербургский Политехнический Университет имени Петра Великого \\ кафедра космических исследований.\\
    \vspace{0.7cm}
    Научный руководитель:  Гладилин Пётр Евгеньевич \\ н.с. ФТИ им. А.Ф.Иоффе, к.ф.-м.н.
    \vspace{0.7cm}
}
\AtBeginSection[] % как ясно из названия, перед каждой секцией вставляем что написано
{
  \begin{frame}{Содержание}
  \tableofcontents[currentsection,hideallsubsections]
  \end{frame}
}
\begin{document}
\begin{frame}
\titlepage 
\end{frame} 
\begin{frame}
\tableofcontents[hideallsubsections] % автоматом получаем содержание (скрываем подсекции)
\end{frame}
\section{Введение}
\begin{frame}{Физика процесса}
\begin{figure}[H]
\center
\includesvg[width=0.60\linewidth]{lamin}
\caption{Ускорение на ударных волнах при ламинарном (а) и турбулентном(б) движении.}
\end{figure}
 
\begin{equation}
n \sim p^{-\gamma}
\end{equation}
 с показателем $\gamma=\frac{\sigma+2}{\sigma-1}$
\end{frame}

\begin{frame}{Исследования}
\begin{itemize}
\item 70-80 гг прошлого века - Аксворд, Лиир, Скадрон, Белл, Крымский, Бережко, Бландфолд, Острикер -- теоретическое исследование проблемы ускорения на ударных волнах
\item 1992 г - Ахтенберг и Круллс - моделирование с использованием стохастического подхода
\item с 80-х годов - моделирование с использованием разностной схемы.
\pause
~\\
\item Никто из авторов не делал количественный анализ двух подходов с точки зрения простоты реализации, затрат компьютерного времени и динамических возможностей.
\end{itemize}
\end{frame}
\section{Цели и задачи}
\begin{frame}{Цели и задачи}
\textit{\texttt{Цель:}} анализ и сравнение 
диффузиозно-конвективного и стохастического подхода в ускорении частиц на ударных волнах.

\textit{\texttt{Задачи:}}
\begin{enumerate}
\item Реализовать численное моделирование процесса ускорения частиц на ударных волнах посредством решения диффузиозно-конвективного уравнения с помощью \texttt{разностной схемы}
\item Реализовать численное моделирование ускорения частиц на ударных волнах \texttt{стохастическим} методом.
\item Провести \texttt{сравнение} данных подходов
\item Указать \texttt{положительные и отрицательные} стороны в каждом из них
\end{enumerate}
\end{frame}

\section{Методы}
\begin{frame}{методы}
\framesubtitle{неявный метод Эйлера}
Изначальное уравнение имеет вид
\begin{equation}
\frac{\partial f}{\partial t} = \frac{\partial}{\partial x} \kappa \frac{\partial f}{\partial x} - u \frac{\partial f}{\partial x} - \frac{du}{3dx} \delta(x) p \frac{\partial f}{\partial p} +Q
\end{equation}

\pause
Разностная схема
\begin{multline}
\frac{f_{i,j,k} - f_{i,j,k-1}}{\Delta t} = \kappa_{j} \frac{f_{i+1,j,k}+f_{i-1,j,k}-2f_{i,j,k}}{\Delta^2 x} \\
- u_i\frac{f_{i,j,k}-f_{i-1,j,k}}{\Delta x}-\frac{u_i-u_{i-1}}{3\Delta x}\frac{f_{i,j,k}-f_{i,j-1,k}}{\Delta y} + Q
\end{multline}
\pause
\begin{itemize}
\item решение трёхдиагональной матрицы методом прогонки
\item язык C++
\end{itemize}
\end{frame}

\begin{frame}{методы}
\framesubtitle{стохастический метод}
Уравнением в виде уравнения Фоккера-Планка
\begin{equation}
\frac{\partial F(\vec{Z}, t)}{\partial t} = \frac{\partial}{\partial \vec{Z}}\left( -\dot{\vec{Z}}F+\frac{\partial}{\partial\vec{Z}}[DF]  \right)
\end{equation}
\pause
Общий вид уравнения:
\begin{equation}
dx = a(x, t)dt + b(x,t) \underbrace{\varepsilon \sqrt{dt}}_{dW}
\end{equation}
\pause
В форме уравнения Ито:
\begin{equation}
d\vec{Z} = d\dot{\vec{Z}}(\vec{Z}, t)+\sqrt{2D}dW
\end{equation}
В приложении к задаче:
\begin{eqnarray}
dx = V(x)dt+\sqrt{2K_{\parallel}}dW\\
du = - \frac{1}{3} \frac{\partial V}{\partial x} dt
\end{eqnarray}
\pause
 язык C++ и Python
\end{frame}

\begin{frame}
\begin{figure}[H]
  \centering
  \includesvg[width=0.70\linewidth]{principle}
  \caption{Последовательность итераций для разностной схемы (слева) и стохастического подхода(справа)}
\end{figure}
\end{frame}

\section{Результаты}
\begin{frame}{Результаты}
\begin{columns}
\begin{column}{0.5\textwidth}
Эйлер
\begin{figure}[H]
\centering
\includegraphics[width=0.90\linewidth]{r_common}
\caption{спектр ускоренных частиц для решения с разностной схемой}
\end{figure}
\end{column}

\begin{column}{0.5\textwidth}
Стохастика
\begin{figure}[H]
\centering
\includegraphics[width=0.90\linewidth]{stoh_bom_one}
\caption{спектр ускоренных частиц для решения стохастическим методом}
\end{figure}
\end{column}
\end{columns}
\end{frame}


\begin{frame}{Результаты}
\framesubtitle{время выполнения}
\begin{columns}
\begin{column}{0.5\textwidth}
Эйлер
\begin{figure}[H]
\centering
\includegraphics[width=0.90\linewidth]{r_times}
\caption{Различные времена запуска}
\end{figure}
\end{column}

\begin{column}{0.5\textwidth}
Стохастика
\begin{figure}[H]
\centering
\includegraphics[width=0.90\linewidth]{stoh_times}
\caption{Различные времена запуска}
\end{figure}
\end{column}
\end{columns}
\end{frame}


\begin{frame}{Результаты}
\framesubtitle{Интенсивность спектра}
\begin{columns}
\begin{column}{0.5\textwidth}
Эйлер
\begin{figure}[H]
\centering
\includegraphics[width=0.90\linewidth]{r_Qinj}
\caption{различные мощности инжекции}
\end{figure}
\end{column}

\begin{column}{0.5\textwidth}
Стохастика
\begin{figure}[H]
\centering
\includegraphics[width=0.90\linewidth]{stoh_particles}
\caption{различное количество частиц}
\end{figure}
\end{column}
\end{columns}
\end{frame}


\begin{frame}{Результаты}
\framesubtitle{виды коэффициента диффузии}
\begin{columns}
\begin{column}{0.5\textwidth}
Эйлер
\begin{figure}[H]
\centering
\includegraphics[width=0.90\linewidth]{r_bom_or_not2}
\caption{различные коэффициенты диффузии}
\end{figure}
\end{column}

\begin{column}{0.5\textwidth}
Стохастика
\begin{figure}[H]
\centering
\includegraphics[width=0.90\linewidth]{stoh_bom_or_not}
\caption{различные коэффициенты диффузии}
\end{figure}
\end{column}
\end{columns}
\end{frame}


\begin{frame}{Результаты}
\framesubtitle{дополнительные результаты для стохастического подхода}
\begin{columns}
\begin{column}{0.5\textwidth}
Стохастика
\begin{figure}[H]
\centering
\includegraphics[width=0.90\linewidth]{stoh_sinh_or_not}
\caption{синхротронные потери}
\end{figure}
\end{column}

\begin{column}{0.5\textwidth}
Стохастика
\begin{figure}[H]
\centering
\includegraphics[width=0.90\linewidth]{stoh_two_or_one}
\caption{две волны}
\end{figure}
\end{column}
\end{columns}
\end{frame}

\begin{frame}{Результаты}
\framesubtitle{сравнение}
\begin{figure}[H]
\centering
\includegraphics[width=0.5\linewidth]{compare}
\caption{Сравнение подходов}
\end{figure}
\end{frame}
\begin{frame}{Результаты}
\framesubtitle{сравнение}
\begin{tabular}{|c|c|c|}
\hline
& Разностная схема & Стохастика \\ \hline
Время & $O(N_tN_pN_x)$ & $O(N_tN)$\\ \hline
Память & $O(N_pN_x)$ & $O(1)$\\ \hline
Параллелизация & по импульсу & по частицам\\ \hline
\end{tabular}
\end{frame}
\section{Выводы}
\begin{frame}{Выводы}
\begin{enumerate}
\item Было получено решение задачи о ускорении частиц на фронте волны с помощью разностной схемы и стохастическим подходом
\item Показана эквивалентность между обоими подходами
\item Проведено сравнение подходов:
\begin{enumerate}
\item Форма графика
\begin{itemize}
\item Форма графика при стохастическом подходе всегда более изломана, нежели при решении с использованием разностной схемы, однако она всегда подчиняется огибающей вида $p^{-1}$, в то время как для метода Эйлера подъём происходит лишь на некоторых больших временах.
\item Для стохастического подхода высота графика неразрывна связана с количеством изломов. Таким образом нельзя получить адекватный результат для небольшого числа частиц. В то время, как при первом подходе высота контролируется отдельным параметром - $Q_{inj}$
\end{itemize}
\item Время работы
\begin{itemize}
\item Время работы программ для одинаковых максимальных импульсов фактически не различается.
\end{itemize}
\item Возможность параллельного выполнения
\begin{itemize}
\item Из сказанного выше следует, что максимальное число потоков, на которое можно разбить программу выше для стохастического подхода.
\item Так же для стохастического подхода, согласно закону Амдала, выше возможное ускорение.
\end{itemize}
\item Возможность расширение кода
\begin{enumerate}
\item Подход с составлением разностных схем более гибок в плане возможности выбора сетки и конкретной схема
\item Стохастический подход более просто в своём основании, что даёт уменьшает время модифицирования программы для различных физических процессов, что можно видеть благодаря "приложению" в результатах
\end{enumerate}
\end{enumerate}
\end{enumerate}
\pause
В обоих подходах есть свои плюсы и минусы, однако по расширяемости стохастический подход проявляет себя лучше подхода с решением с помощью разностных схем.
\end{frame}

\begin{frame}

\begin{figure}[H]
\centering \Large
Спасибо за внимание
\includegraphics[width=0.8\linewidth]{shock_ill}
\end{figure}
{\small (изображение с \url{http://i.ytimg.com/})}
\end{frame}
\end{document}