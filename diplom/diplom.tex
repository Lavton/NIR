\documentclass[a4paper,11pt]{article} %размер бумаги устанавливаем А4, шрифт 12пунктов
\usepackage[T2A]{fontenc}
%\usepackage[utf8]{inputenc} 
\usepackage[utf8x]{inputenc}
\usepackage[T1]{fontenc}
\usepackage{textcomp}
\inputencoding{utf8}
\usepackage[english,russian]{babel}%используем русский и английский языки с переносами
\usepackage{amssymb,amsfonts,amsmath,mathtext,cite,enumerate,float}%подключаем нужные пакеты расширений
\usepackage[pdftex]{graphicx} %хотим вставлять в диплом рисунки?
\usepackage{color}
\usepackage[ section ]{ placeins}
\usepackage[pdftex]{lscape}
\usepackage{lscape}
%\graphicspath{{noiseimages/}}%путь к рисункам
\usepackage{listings}
\lstset{language=Fortran}
\makeatletter
\renewcommand{\@biblabel}[1]{#1.} % Заменяем библиографию с квадратных скобок на точку:
\makeatother
\setcounter{tocdepth}{3}
\usepackage{geometry} % Меняем поля страницы
\makeatletter
\@addtoreset{equation}{subsection}
\makeatother

\newcommand{\frp}[2]{\frac{\partial #1}{\partial #2}}

\renewcommand*{\thesection}{\arabic{section}}

\geometry{left=2cm}% левое поле
\geometry{right=1.5cm}% правое поле
\geometry{top=1cm}% верхнее поле
\geometry{bottom=2cm}% нижнее поле



\renewcommand{\theequation}{\arabic{subsection}.\arabic{equation}}

\renewcommand{\theenumi}{\arabic{enumi}}% Меняем везде перечисления на цифра.цифра
\renewcommand{\labelenumi}{\arabic{enumi}}% Меняем везде перечисления на цифра.цифра
\renewcommand{\theenumii}{.\arabic{enumii}}% Меняем везде перечисления на цифра.цифра
\renewcommand{\labelenumii}{\arabic{enumi}.\arabic{enumii}.}% Меняем везде перечисления на цифра.цифра
\renewcommand{\theenumiii}{.\arabic{enumiii}}% Меняем везде перечисления на цифра.цифра
\renewcommand{\labelenumiii}{\arabic{enumi}.\arabic{enumii}.\arabic{enumiii}.}% Меняем везде перечисления на цифра.цифра
\renewcommand{\theequation}{\arabic{equation}}
\begin{document}

\tableofcontents
\newpage
\section{Введение}
\subsection{Актуальность}
В течение продолжительного времени учёные пытаются определить, как образуется наблюдаемое распределение космических лучей по энергиям. Далеко не весь диапазон спектра можно описать с помощью теплового взаимодействия.

Особую роль играют механизмы нетеплового ускорения частиц в бесстолкновительной космической плазмы. 

Такие частицы можно непосредственно наблюдать в межпланетном пространстве, частицы низких энергий - непосредственно наблюдать с Земли, а частицы высоких энергий ($>10^5$ ГэВ) - в виде широких атмосферных ливней.

Особый интерес составляет процесс на фронтах ударных волн, в силу их распространённости.
Примерами таких волн могут служить волны от хромосферических солнечных вспышек, вспышек сверхновых звёзд...

Дополнительный интерес к данным процессам связан с возможностью выделения в них большого количества энергии, существенная часть которой может быть преобразована в направленное ускорение небольшого количества частиц, что и приводит к появлению частиц с энергиями, на много порядков превышающих тепловые.

Понимание данных процессов необходимы для построения моделей ударных волн.

Проблемой занимаются давно.\\
 Бережко Е. Г. и Крымский Г. Ф. в 1988 году привели теоретический вывод диффузиозно-конвективного уравнения. Х. Канг в 2011 году провёл численные расчёты для этого вида. \\ 
С другой стороны к задаче подошёли А. Ахтенберг и В. Круллс, которые в 1992 году провёли рассмотрение ускорения стохастическим подходом. 

Однако ни один из этих авторов не пытался сравнивать оба подхода.

\subsection{Цели и задачи}
В связи со всем вышесказанным целью работы был анализ и сравнение диффузиозно-конвективного и стохастического подхода в ускорении частиц на ударных волнах.

В рамках этой цели были поставлены следующие задачи:
\begin{enumerate}
\item Реализовать численное моделирование диффузиозно-конвективного подхода в ускорении частиц на ударных волнах.
\item Реализовать численное моделирование стохастического процесса в ускорении частиц на ударных волнах.
\item Провести сравнение данных подходов
\item Указать положительные и отрицательные стороны в каждом из них
\end{enumerate}

\subsection{Методы исследования}
Для решения диффузиозно-конвективного уравнения была разработана программа на языке C++, организующая численное моделирование методом прогонки.

Стохастическое уравнение решалось в виде уравнения Ито и реализацией алгоритма так же на языке C++.

Остановимся подробнее на каждом из них
\subsubsection{Моделирование диффузиозно-конвективного уравнения методом прогонки}
Изначальное уравнение имеет вид
\begin{equation}
\frac{\partial f}{\partial t} = \frac{\partial}{\partial x} \kappa \frac{\partial f}{\partial x} - u \frac{\partial f}{\partial x} - \frac{\Delta u}{3} \delta(x) p \frac{\partial f}{\partial p} +Q
\end{equation}

Решим данное уравнение с помощью неявного метода Эйлера.

Для этого надо подставить вместо производных конечные разности и получить следующий общий вид:
\begin{equation}
\frac{f_{i,j,k} - f_{i,j,k-1}}{\Delta t} = \kappa_{j} \frac{f_{i+1,j,k}+f_{i-1,j,k}-2f_{i,j,k}}{\Delta^2 x} - u_i\frac{f_{i,j,k}-f_{i-1,j,k}}{\Delta x}-\frac{u_i-u_{i-1}}{3\Delta x}\frac{f_{i,j,k}-f_{i,j-1,k}}{\Delta y} + Q
\end{equation}
Далее система записывается в виде матричного уравнения
\begin{equation}
Mx=F
\end{equation}
где $M$ - трёхдиаганальная матрица с $A_i$ на поддиаганали, $C_i$ - на главной диаганали и $B_i$ на наддиаганали. В приложении задачи коэффициенты получились следующими:
\begin{eqnarray}
A_i&=\frac{\Delta t \kappa_{i,j}}{\Delta^2 x} + \frac{u_i\Delta t}{\Delta x}\\
C_i&=\frac{2\Delta t \kappa_{i,j}}{\Delta x^2} + \frac{\Delta t u_i}{\Delta x} + \frac{u_i-u_{i-1}\Delta t}{3\Delta x \Delta y} - 1\\
B_i&=\frac{\Delta t \kappa_{i,J}}{\Delta^2 x}\\
F_i&=-f_{i,j,k-1}-\frac{f_{i,j-1,k}}{\Delta y} \frac{u_i-u{i-1}\Delta t}{3\Delta x} - Q\Delta t
\end{eqnarray} 
Далее данное уравнение решается методом прогонки.
\subsubsection{Моделирование стохастического процесса}
В данном случае уравнение принимает вид
\begin{equation}
\frac{\partial F(\vec{Z}, t)}{\partial t} = \frac{\partial}{\partial \vec{Z}}\left( -\dot{\vec{Z}}F+\frac{\partial}{\partial\vec{Z}}[DF]  \right)
\end{equation}
Где $Z\equiv (x, p)$.
\begin{equation}
\dot{\vec{Z}} = \left< \frac{d\vec{Z}}{dt} \right> + \frac{\partial}{\partial\vec{Z}} D
\end{equation}
Это уравнение может быть представлено в форме уравнения Ито:
\begin{equation}
d\vec{Z} = d\dot{\vec{Z}}(\vec{Z}, t)+\sqrt{2D}dW
\end{equation}
Где $dW$ - Винеровский процесс
В приложении к конкретной задачи для заряженных частиц имеем
\begin{eqnarray}
dx = V(x)dt+\sqrt{2K_{\parallel}}dW\\
du = - \frac{1}{3} \frac{\partial V}{\partial x} dt
\end{eqnarray}
Как известно, общий вид уравнения Ито выглядит как
\begin{equation}
dx = a(x, t)dt + b(x,t) \underbrace{\varepsilon \sqrt{dt}}_{dW}
\end{equation}
где $a$ имеет физический смысл сноса частицы, а $b$ - диффузии. $\varepsilon$ - нормально распределённая случайная величина.

Задав обе функции, а так же размер и количество шагов по времени, мы получаем предсказание координаты и импульса для одной частицы.

Запустив код для $N$ частиц, можно получить массивы координат и импульсов для конца времени. Далее строя гистограмму распределения для импульса мы получаем империческую функцию распределения $f_p$, которая может быть сравнима с аналогичной функцией, полученной в другом подходе.

Как было написано выше, $a(x,t) \equiv V(x)$, где $V(x)$ - скорость для ударной волны. Стандартная степень сжатия на фронте равна 4, т.е. $V|_{x \ll x_{inj}} = 4V|_{x \gg x_{inj}}$
Для устойчивости работы программы должно выполняться условие
\begin{equation}
V\Delta t \le \Delta x_s \ll \sqrt{K_\parallel \Delta t}
\end{equation}
Где $\Delta x_s$ - характерное масштаб фронта. ~\\~\\
В качестве демонстрации преимуществ данного подхода были реализовано ещё два физических процесса: ускорение электронов с учётом синхротронных потерь и ускорение на сходящихся фронтах.\\
Синхротронные потери можно ввести, заменив уравнение для логарифма импульса на 
\begin{equation}
du = - \left( \frac{1}{3} \frac{\partial V}{\partial x}  + \beta_s\sqrt{1+e^{2u}} \right) dt
\end{equation}
Ускорение на сходящихся фронтах соответствует занулению скорости вдали от фронта и увеличению числа частиц в двое.
%Численное моделирование ускорения частиц на ударных волнах
\bibliographystyle{gost71s}
\bibliography{Biblio}

\end{document}
