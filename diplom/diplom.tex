\documentclass[a4paper,11pt]{article} %размер бумаги устанавливаем А4, шрифт 12пунктов
\usepackage[T2A]{fontenc}
%\usepackage[utf8]{inputenc} 
\usepackage[utf8x]{inputenc}
\usepackage[T1]{fontenc}
\usepackage{textcomp}
\inputencoding{utf8}
\usepackage[english,russian]{babel}%используем русский и английский языки с переносами
\usepackage{amssymb,amsfonts,amsmath,mathtext,cite,enumerate,float}%подключаем нужные пакеты расширений
\usepackage[pdftex]{graphicx} %хотим вставлять в диплом рисунки?
\usepackage{color}
\usepackage[ section ]{ placeins}
\usepackage[pdftex]{lscape}
\usepackage{lscape}
%\graphicspath{{noiseimages/}}%путь к рисункам
\usepackage{listings}
\lstset{language=Fortran}
\makeatletter
\renewcommand{\@biblabel}[1]{#1.} % Заменяем библиографию с квадратных скобок на точку:
\makeatother
\setcounter{tocdepth}{3}
\usepackage{geometry} % Меняем поля страницы
\makeatletter
\@addtoreset{equation}{subsection}
\makeatother

\newcommand{\frp}[2]{\frac{\partial #1}{\partial #2}}

\renewcommand*{\thesection}{\arabic{section}}

\geometry{left=2cm}% левое поле
\geometry{right=1.5cm}% правое поле
\geometry{top=1cm}% верхнее поле
\geometry{bottom=2cm}% нижнее поле



\renewcommand{\theequation}{\arabic{subsection}.\arabic{equation}}

\renewcommand{\theenumi}{\arabic{enumi}}% Меняем везде перечисления на цифра.цифра
\renewcommand{\labelenumi}{\arabic{enumi}}% Меняем везде перечисления на цифра.цифра
\renewcommand{\theenumii}{.\arabic{enumii}}% Меняем везде перечисления на цифра.цифра
\renewcommand{\labelenumii}{\arabic{enumi}.\arabic{enumii}.}% Меняем везде перечисления на цифра.цифра
\renewcommand{\theenumiii}{.\arabic{enumiii}}% Меняем везде перечисления на цифра.цифра
\renewcommand{\labelenumiii}{\arabic{enumi}.\arabic{enumii}.\arabic{enumiii}.}% Меняем везде перечисления на цифра.цифра
\renewcommand{\theequation}{\arabic{equation}}
\begin{document}

\tableofcontents
\newpage
\section{Введение}
\subsection{Актуальность}
В течение продложительного времени учёные пытаются определить, как образуется наблюдаемое распределение космических лучей по энергиям. Далеко не весь диапазон спектра можно описать с помощью теплового взаимодействия.

Особую роль играет бесстолкновительное ускорение плазмы на фронтах ударных волн. Однако, на данный момент не представляется возможным прямая регистрация распределения частиц по энергиям и эту нишу занимает численное моделирование

Проблемой занимаются давно.\\
 Бережко Е. Г. и Крымский Г. Ф. в 1988 году опубликовали теоретический вывод диффузиозно-конвективного уравнения. Х. Канг в 2011 году провёл численные расчёты для этого вида. \\ 
С другой стороны к задаче подошёл А. Круллс, который в 1992 году провёл рассмотрение ускорения стохастическим методом. 

Однако ни один из этих авторов не пытался сравнивать оба подхода и развивать их в ту сторону, в которую это возможно.

\subsection{Цели и задачи}
В связи со всем вышесказанным целью работы было анализ и сравнение диффузиозно-конвективного и стохастического подхода.

В рамках этой цели были поставлены следующие задачи
\begin{enumerate}
\item Реализовать численное моделирование диффузиозно-конвективного подхода
\item Реализовать численное моделирование стохастического процесса
\item Провести сравнение данных подходов
\item Указать положительные и отрицательные стороны в каждом из них
\end{enumerate}

\subsection{Методы исследования}
Для решения диффузиозно-конвективного уравнения была разработана программа на языке C++, организующая численное моделирование методом прогонки.

Стохастическое уравнение решалось в виде уравнения Ито и реализацией алгоритма так же на языке C++.

%Численное моделирование ускорения частиц на ударных волнах


\end{document}
